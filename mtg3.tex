% \documentclass[dvipdfmx, 11pt]{beamer}
\documentclass[aspectratio=169, dvipdfmx, 11pt]{beamer} % aspectratio=43, 149, 169
\usepackage{here, amsmath, latexsym, amssymb, bm, ascmac, mathtools, multicol, tcolorbox, subfig}

%デザインの選択(省略可)
\usetheme{Luebeck}
%カラーテーマの選択(省略可)
\usecolortheme{orchid}
%フォントテーマの選択(省略可)
\usefonttheme{professionalfonts}
%フレーム内のテーマの選択(省略可)
\useinnertheme{circles}
%フレーム外側のテーマの選択(省略可)
\useoutertheme{infolines}
%しおりの文字化け解消
\usepackage{atbegshi}
\ifnum 42146=\euc"A4A2
\AtBeginShipoutFirst{\special{pdf:tounicode EUC-UCS2}}
\else
\AtBeginShipoutFirst{\special{pdf:tounicode 90ms-RKSJ-UCS2}}
\fi
%ナビゲーションバー非表示
\setbeamertemplate{navigation symbols}{}
%既定をゴシック体に
\renewcommand{\kanjifamilydefault}{\gtdefault}
%タイトル色
\setbeamercolor{title}{fg=structure, bg=}
%フレームタイトル色
\setbeamercolor{frametitle}{fg=structure, bg=}
%スライド番号のみ表示
%\setbeamertemplate{footline}[frame number]
%itemize
\setbeamertemplate{itemize item}{\small\raise0.5pt\hbox{$\bullet$}}
\setbeamertemplate{itemize subitem}{\tiny\raise1.5pt\hbox{$\blacktriangleright$}}
\setbeamertemplate{itemize subsubitem}{\tiny\raise1.5pt\hbox{$\bigstar$}}
% color
\newcommand{\red}[1]{\textcolor{red}{#1}}
\newcommand{\green}[1]{\textcolor{green!40!black}{#1}}
\newcommand{\blue}[1]{\textcolor{blue!80!black}{#1}}

%----------------------------------↑設定--------------------------------%

\title{総合研究における構想}
\author[Ota Kawai]{BV21053 河合桜汰}
\institute[G-Zhai Lab]{協調制御研究室}
\date{January 18,2024}

\begin{document}

\frame{
\titlepage
}

\frame{
\frametitle{発表概要}
\tableofcontents
}

\section{今までの学習内容}
\frame{\tableofcontents[currentsection]} %%目次の参照
\begin{frame}
\frametitle{今までの学習内容}%---
\begin{center}
\begin{itemize}
    \item \textbf{グラフ理論}(ex.完全グラフ、次数、同型)
    \vskip \baselineskip
    \item \textbf{木やスタック、キュー、幅優先探索などのアルゴリズム}(ex.全域木、2分探索木)
    \vskip \baselineskip
    \item \textbf{経路最適化}(ex.ダイクストラ法、郵便配達人問題、巡回セールスマン問題、最大流問題)
    \vskip \baselineskip
    \item \textbf{グラフの分割と最適化}(ex.マッチング、時間割問題、4色定理)
\end{itemize}
\end{center}
\end{frame}

\section{興味を持ったこと}
\frame{\tableofcontents[currentsection]} %%目次の参照

\begin{frame}
\frametitle{興味を持ったこと}%---
経路最適化に興味を持った。
\vskip \baselineskip
→元々カーナビを作成したいと感じていた。その中で経路最適化が非常に重要であると感じたため。
\end{frame}

\section{総合研究について}
\frame{\tableofcontents[currentsection]} %%目次の参照

\begin{frame}
\frametitle{総合研究について}
\textbf{総合研究のGOAL(目標):実際にスマホで使用できるカーナビを作成する}
\vskip \baselineskip
研究に必要になるもの
\vskip \baselineskip
\begin{itemize}
    \item 地図データと位置情報
    \vskip \baselineskip
    \item 位置情報処理
    \vskip \baselineskip
    \item フロントエンド、バックエンド
    \vskip \baselineskip
    \item データベース
    \vskip \baselineskip
    \item サーバー
\end{itemize}
\end{frame}

\begin{frame}
\frametitle{総合研究について}
\vskip \baselineskip
\textbf{\textbf{〜技術スタック(開発言語やサーバー)について〜}}
\vskip \baselineskip
アプリの外観(フロントエンド)→TypeScript
\vskip \baselineskip
カーナビの実装アルゴリズム(バックエンド)→Python
\vskip \baselineskip
スマホアプリの作成→ReactNative
\vskip \baselineskip
アプリを動かすためのサーバー(インフラ)→Amazon Web Service
\includegraphics[width=0.3\linewidth]{img/typescript.png}
\includegraphics[width=0.3\linewidth]{img/ReactNative.png}
\includegraphics[width=0.3\linewidth]{img/aws.png}
\vspace{\baselineskip}
\end{frame}

\section{今後の研究計画}
\frame{\tableofcontents[currentsection]} %%目次の参照

\frame{
\frametitle{今後の研究計画}%---
春休みの大まかな予定:
\vskip \baselineskip
1月下旬〜2月末:ReactNativeのキャッチアップ、Pythonの深い理解
\vskip \baselineskip
3月〜3月末:AWSでのサーバー構築、Pythonを用いた経路最適化の勉強
\vskip \baselineskip
4月〜未定:カーナビで必要なデータの洗い出し、小さい範囲でのカーナビの作成
}

\section{参考文献}
\frame{\tableofcontents[currentsection]} %%目次の参照

\frame{
\frametitle{参考文献}%---
Pythonによる数理最適化入門【朝倉書店】
 \vskip \baselineskip
新・明解Python入門【=SB Creative】
\vskip \baselineskip
NAVITIME JAPANテックブログ
\vskip \baselineskip
Android Developersブログ
}


\end{document}