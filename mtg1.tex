% \documentclass[dvipdfmx, 11pt]{beamer}
\documentclass[aspectratio=169, dvipdfmx, 11pt]{beamer} % aspectratio=43, 149, 169
\usepackage{here, amsmath, latexsym, amssymb, bm, ascmac, mathtools, multicol, tcolorbox, subfig}

%デザインの選択(省略可)
\usetheme{Luebeck}
%カラーテーマの選択(省略可)
\usecolortheme{orchid}
%フォントテーマの選択(省略可)
\usefonttheme{professionalfonts}
%フレーム内のテーマの選択(省略可)
\useinnertheme{circles}
%フレーム外側のテーマの選択(省略可)
\useoutertheme{infolines}
%しおりの文字化け解消
\usepackage{atbegshi}
\ifnum 42146=\euc"A4A2
\AtBeginShipoutFirst{\special{pdf:tounicode EUC-UCS2}}
\else
\AtBeginShipoutFirst{\special{pdf:tounicode 90ms-RKSJ-UCS2}}
\fi
%ナビゲーションバー非表示
\setbeamertemplate{navigation symbols}{}
%既定をゴシック体に
\renewcommand{\kanjifamilydefault}{\gtdefault}
%タイトル色
\setbeamercolor{title}{fg=structure, bg=}
%フレームタイトル色
\setbeamercolor{frametitle}{fg=structure, bg=}
%スライド番号のみ表示
%\setbeamertemplate{footline}[frame number]
%itemize
\setbeamertemplate{itemize item}{\small\raise0.5pt\hbox{$\bullet$}}
\setbeamertemplate{itemize subitem}{\tiny\raise1.5pt\hbox{$\blacktriangleright$}}
\setbeamertemplate{itemize subsubitem}{\tiny\raise1.5pt\hbox{$\bigstar$}}
% color
\newcommand{\red}[1]{\textcolor{red}{#1}}
\newcommand{\green}[1]{\textcolor{green!40!black}{#1}}
\newcommand{\blue}[1]{\textcolor{blue!80!black}{#1}}

%----------------------------------↑設定--------------------------------%

\title{Pythonによるグラフ最適化}
\author[Ota Kawai]{BV21053 河合桜汰}
\institute[G-Zhai Lab]{協調制御研究室}
\date{November 30,2023}

\begin{document}

\frame{
\titlepage
}

\frame{
\frametitle{発表概要}
\tableofcontents
}

\section{現在まで勉強したこと}
\frame{\tableofcontents[currentsection]} %%目次の参照

\frame{
\frametitle{現在まで勉強したこと}%---
Pythonというプログラミング言語を用いて、グラフ最適化の勉強をしてきました。現在は、様々なアルゴリズムや問題をPythonで実装しながら勉強を進めています。
また、Pythonについても初めて触れたので文法書も並行して進めています。

 \vspace{\baselineskip}
(例)木構造、ネットワーク連結問題、ダイクストラ法...
}

\section{グラフ最適化とは}
\frame{\tableofcontents[currentsection]} %%目次の参照

\frame{
\frametitle{グラフ最適化とは}%---

グラフ最適化は、グラフ理論の枠組みで構築された問題を解く手法やアルゴリズム。
 \vskip \baselineskip
(グラフ最適化の応用例)
 \vskip \baselineskip
最短経路の探索: ある2つの都市間の最短経路を見つける。
 \vskip \baselineskip
最小費用の経路探索: 移動や通信にかかるコストが最小の経路を見つける。
 \vskip \baselineskip
最大流量の探索: グラフ内の経路において最大の流量を見つける。
}
\frame{
\frametitle{グラフ最適化とは}%---
グラフ最適化には様々なアルゴリズムがある。
 \vskip \baselineskip
・ダイクストラ法
 \vskip \baselineskip
・最小全域木アルゴリズム
 \vskip \baselineskip
・深さ優先探索(DFS)や幅優先探索(BFS)
}

\section{グラフの数式による表現と様々なグラフ}
\frame{\tableofcontents[currentsection]} %%目次の参照

\begin{frame}
\frametitle{グラフの数式による表現}
\begin{block}{グラフの数式表現}
  G = (V,E),\quad V = \{A,B,C,D\},\quad E = \{a,b,c,d,e,f,g\}
  \quad a = \{A,B\},\quad b = \{A,B\}, \quad c = \{A,C\}
\end{block}
\includegraphics[width=1.0\linewidth]{konigsberg.png}
\end{frame}

\begin{frame}
\frametitle{様々なグラフ}
\begin{block}{様々なグラフ}
自己ループや多重辺を持たないグラフを単純グラフ(simple graph)という。さらにどの頂点間も枝で隣接している単純グラフを完全グラフ(complete graph)といい、頂点が$n$の完全グラフを$K_n$で表す。
\end{block}
\centering
\includegraphics[width=0.5\linewidth]
{completeGraph.png}
\end{frame}

\begin{frame}
\frametitle{様々なグラフ}
\begin{block}{様々なグラフ}
グラフの頂点集合$V$が2つの部分集合$X$と$Y$に分割され、どの枝も$X$と$Y$の点両方に接続しているとき、そのグラフを2部グラフ(bipartite graph)といいます。さらに$X$と$Y$のどの頂点間にも枝が存在する時、完全2部グラフといい、$K_{m,n}$で表します。ただし、$m$を頂点の数、$n$を$Y$の頂点数とします。ここで、$V$が$X$と$Y$に分割されるとは、$V = X \cup Y$かつ$X \cap Y = \emptyset$が成り立つことである。
\end{block}
\centering
\includegraphics[width=0.4\linewidth]
{twoCompleteGraph.png}
\end{frame}


\section{今後の研究計画}
\frame{\tableofcontents[currentsection]} %%目次の参照

\frame{
\frametitle{今後の研究計画}%---
グラフの分割と最適化について学ぶ。
 \vskip \baselineskip
 その中でも、辺彩色問題や時間割問題について詳しく学習する。
}

\section{参考文献}
\frame{\tableofcontents[currentsection]} %%目次の参照

\frame{
\frametitle{参考文献}%---
Pythonによる数理最適化入門【朝倉書店】
 \vskip \baselineskip
新・明解Python入門【=SB Creative】
}


\end{document}